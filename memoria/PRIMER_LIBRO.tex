\documentclass[12pt,a4paper,Spanish]{book}
%Va a ser un libro (book), el tamaño es a4, la lengua castellano%%%

\usepackage[spanish]{babel} %palabras de multitud de idiomas. Aquí no se si 
\usepackage[utf8]{inputenc} %cambio latin por utf-8
\usepackage{amsmath} %macros AM
\usepackage{amsthm} %macros AMS para teoremas.
\usepackage{amsfonts} %Permite usar fuentes.
%\usepackage[dvips]{epsfig} %Inclusión de figuras postscript
\usepackage{indentfirst}
\usepackage{graphicx}
\usepackage{subfigure}
\usepackage{hyperref}
\setlength{\parskip}{8mm} %Separación entre parrafos 
\usepackage{multirow, array} % para las tablas
%\usepackage{longtable} % para tablas largas
\usepackage{afterpage}
\usepackage[left=3 cm,top=2.5cm,right=3cm,bottom=2.5cm]{geometry} 
\usepackage{cite} % para contraer referencias

\DeclareGraphicsExtensions{.jpg, .pdf, .png, .gif, .eps}



\begin{document}

\renewcommand{\listtablename}{Índice de tablas} 

%%PORTADA%%

\begin{titlepage}
\begin{center}


\begin{figure}
	\centering
\includegraphics[width=0.3\textwidth]{./figs/logoURJC}
\end{figure}

\begin{center}
\large
ESCUELA DE INGENIERÍA DE FUENLABRADAa

\vspace*{0.15in}
GRADO EN INGENIERÍA DE SISTEMAS AUDIOVISUALES Y MULTIMEDIA \\
\vspace*{0.6in}




{\large \bf TRABAJO FIN DE GRADO}\\
\end{center}
\vspace*{0.2in}
{\large
{TÍTULO} \\
}
\vspace*{0.3in}
\vspace*{0.3in}

\vspace*{0.1in}
\end{center}

{\large
Autor:   \\[0.2cm]

Tutora: \\[0.15cm]

}
\vspace*{0.1in}
\vspace*{0.1in}

\begin{center}Curso académico 2023/2024\end{center}



\end{titlepage}
%%%%%%%%%%%%%%%%%%%%%%%%%
\newpage
$\ $
\thispagestyle{empty} % para que no se numere esta pagina

%%%%%%%%%%%%%%%%%%%%%dedicatoria%%%%%%%
\chapter*{}
\pagenumbering{Roman} % para comenzar la numeracion de paginas en numeros romanos
\begin{flushright}
\textit{“bla bla" ()}

\end{flushright}
%%%%%%%%%%%%%%%%%%%%%%%%%

\chapter*{Agradecimientos} % si no queremos que añada la palabra "Capitulo"
\addcontentsline{toc}{chapter}{Agradecimientos} % si queremos que aparezca en el índice
\markboth{AGRADECIMIENTOS}{AGRADECIMIENTOS} % encabezado 

AGRADECIMIENTOS 

prueba

\chapter*{Resumen}
\addcontentsline{toc}{chapter}{Resumen} % si queremos que aparezca en el índice

VAS ESCRIBIENDO
\tableofcontents



\cleardoublepage
\addcontentsline{toc}{chapter}{Lista de figuras} % para que aparezca en el indice de contenidos
\listoffigures % indice de figuras

\cleardoublepage
\addcontentsline{toc}{chapter}{Lista de tablas} % para que aparezca en el indice de contenidos
\listoftables % indice de tablas



%\listoffigures
%\listoftables



\chapter{INTRODUCCION}\label{Introduccion}
\pagenumbering{arabic} % para empezar la numeración con números



\section{Motivación}


\subsection{título}


vkkkkk



\section{Objetivos} 

OBJETIVOS

\begin{itemize}
\item OB1 jjjjj	

\item  en la figure

\item OB3
\end{itemize}

\section{Metodología y estructura de la memoria}

Metodología:

\begin{itemize}

\item Revisión bibliográfica ...

\item Aprendizaje de ...
\item Análisis exploratorio de la base de datos...\cite{veterinaria}

\item ...

Estructura de la memoria:

\item En el Capítulo ...

\item ...

\item ...

\item ...

\end{itemize}

\chapter{Estado del arte}\label{Estado del arte}
blalba \cite{yang2006music}
\section{ECG}



\section{VFC,SampEn,MSE,TI}
\section{FA}
\section{Técnicas de ML}


\subsection{Naive Bayes}\label{Modelo OLS}

\subsection{...}

\subsection{Aspectos prácticos}
Normalización datos, Entrenamiento, Validacion...
\chapter{Descripción de datos}


\chapter{Experimentos}


\section{1}
\section{2}





%------------------------------------------------------------------------------------------------
%Chapter Resultados
%------------------------------------------------------------------------------------------------
\chapter{Resultados}\label{Analisis exploratorio}



\section{Experimento 1}

En la Tabla....

\section{Experimento 2} 

\section{....} 




\chapter{Conclusiones}\label{Conclusiones}




%%%%%BIBLIOGRAFÍA%%%%%%%%%%%%%%%
\begin{thebibliography}{99}

\bibitem{veterinaria} \textsc{Publicación complejo respiratiorio porcino}  \textit{http://www.webveterinaria.com}
	
\bibitem{yang2006music}\textsc{
YANG, Yi-Hsuan; LIU, Chia-Chu; CHEN, Homer H. Music emotion classification: A fuzzy approach. En Proceedings of the 14th ACM international conference on Multimedia. 2006. p. 81-84.
}

\end{thebibliography}

\end{document}
